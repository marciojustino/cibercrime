\chapter{Conclus�o}

Como podemos ver ao longo do documento o cibercrime\footnote{� todo crime onde o computador � o principal meio para comet�-lo} tem diversas facetas e muda constantemente. Dentro deste, a fraude � um dos principais crimes cometidos e tem como maior vetor o phishing, inclusive foi apresentado um caso recente que chamou a aten��o da m�dia recentemente \footnote{Caso Carolina Dieckmann}.

Nesse problema mundial de fraudes, o Brasil se destaca negativamente ao liderar o \emph{ranking} dos maiores produtores e utilizadores de \emph{banker} do mundo e que ao contr�rio do que se possa esperar a maioria dos ataques feitos s�o realizados por pessoas com pouco conhecimento t�cnico que se utilizam de \emph{receitas de bolos\footnote{C�digos maliciosos de terceiros}} para a pr�tica do il�cito.

Podemos analisar a estrutura atual do crime organizado e como o processo funciona aliciando laranjas para mascarar o dinheiro furtado. Analisamos quest�es legais e chegamos a discurs�o sobre preven��o e combate ao cibercrime em diversos aspectos e seus novos desafios. Mostramos a rea��o do mercado com seus produtos e solu��es comerciais. Por fim mostramos o estrago que o cibercrime causa na economia.