\chapter{Case de sucesso}

\section{OTP (One Time Password) com integra��o ao Connectra da Check Point}

A implementa��o da ferramenta em conjunto assegurou uma seguran�a maior e resultou na redu��o � quase zero da incid�ncia de fraudes.

A companhia implementou essa ferramenta em uma rede de loja varejista que comercializava seus produtos e servi�os atrav�s de canais de venda distribu�dos pelo Brasil. Os sistemas de frente de caixa e retaguarda eram disponibilizados aos canais pela web, o que havia alto �ndice de fraudes. ``O grande problema nesse caso era o compartilhamento de senhas e a gest�o desse ambiente''. 

``Foi usada a integra��o do Connectra com o OTP para proporcionar autentica��o segura com custo baixo. Tamb�m foi implementado um sistema de gest�o de identidades para sincronizar os contratos do lojista e conceder os acessos em tempo real. 

Nosso cliente tem acesso a uma p�gina na web onde est�o os aplicativos e essa pagina baixa um script que verifica a seguran�a da esta��o. Em um segundo momento, o usu�rio envia as credenciais (conta e senha) mais a OTP, onde � garantida a seguran�a de quem est� acessando estabelecendo um canal seguro. Com o OTP conseguimos integrar as aplica��es com efeito na execu��o e combate a fraudes, roubo de senhas foram evitados nesse processo'' (Julio Graziano Pontes - Gerente de Servi�os da True Access)
