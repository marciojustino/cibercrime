\chapter{Arquitetura Tecnol�gica}

Pela pr�pria natureza do cibercrime ser diversa e abrangente, os equipamentos utilizados para a sua pr�tica s�o t�o diversos e amplos quanto. Mesmo que os computadores sejam o principal meio, nada impede que tablets, smartphones, roteadores etc sejam utilizados de alguma forma como meio para cometer o crime cibern�tico. 

De forma an�loga as tecnologias empregadas pelos equipamentos ou dispon�veis para estes tamb�m s�o abrangentes. Segundo an�lises da Kaspersky Lab \cite{KAS} � comum uma especializa��o de c�digos maliciosos por regi�o. Nos EUA o c�digo malicioso mais comum s�o do tipo FAKEAV\footnote{Falsos antivirus}, na Europa oriental, Russia e pa�ses ib�ricos s�o mais comuns rootkits, o Brasil entretando se especializou na produ��o de c�digos maliciosos para furto de dados banc�rios e clones de cart�o de cr�ditos tamb�m conhecidos como bankers.

Como j� foi dito o Brasil foi classificado como pa�s l�der em v�rus que roubam dados banc�rios, tamb�m conhecido como trojans bankers, segundo pesquisa da Kaspersky Lab \cite{DWKS}, sendo o mesmo o 95\% dos c�digos maliciosos desenvolvidos no Brasil s�o justamente para esse intuito \cite{ADBV}.